%\documentclass[fleqn]{scrartcl}
\documentclass{scrartcl}
\usepackage{amsmath,lscape}
%\setlength{\mathindent}{0pt}
\newcommand{\VHC}{c_\mathrm{VHC}}
\addtolength{\textheight}{1in}

\title{Theory for derivation of the expressions for Douglas-Gunn Alternating Direction Implicit (DG-ADI) method for solving the inhomogeneous heat diffusion equation on a rectangular cuboid finite element grid}
\author{Anders K. Hansen}

\begin{document}

\maketitle

\section{Heat equation}
$t$ is the time. The remaining quantities are considered functions of space: The volumetric heat capacity $\VHC$, with units of $\mathrm{\frac J {m^3 K}}$, the temperature $T$ with units of $\mathrm K$, the rate of heat deposition per unit volume per unit time $q$ with units of $\mathrm{\frac W {m^3 s}}$ and the thermal conductivity $k$, with units of $\mathrm{\frac W {m \cdot K}}$.

The heat diffusion equation is
\begin{equation}
\VHC \frac{dT}{dt} = \nabla \cdot \left(k \nabla T\right) + q
\end{equation}

\section{Discretization in 1D}
Let $k_{x+}$ denote the effective thermal conductivity for heat flow between a voxel and its neighbor in the plus x direction: $k_{x_+} = \frac{2k_ik_{i+1}}{k_i + k_{i+1}}$. Similarly, $k_{x_-} = \frac{2k_{i-1}k_i}{k_{i-1} + k_i}$.

Writing the equation in 1D yields
\begin{equation} \label{1Dcont}
\VHC \frac{dT}{dt} = \frac d {dx} \left( k \frac d {dx} T \right) + q
\end{equation}
Using the notation $T^n$ to denote the temperature at time step $n$, we can discretize the above as
\begin{equation}
\VHC \frac{\Delta T}{\Delta t} = \frac{\Delta_x}{\Delta x} \left( k \frac{\Delta_x}{2\Delta x} \left( T^n + T^{n+1} \right) \right) + q
\end{equation}
where we have chosen to use the mean $\frac 1 2 \left( T^n + T^{n+1} \right)$ as the estimate of the temperature.

We can expand this to get
\begin{multline}
\VHC \frac{T^{n+1}-T^n}{\Delta t} = \frac 1 {2\Delta_x x^2} \Delta_x \left( k \Delta \left(T^n + T^{n+1} \right) \right) + q\\
= \frac 1 {2\Delta x^2} \left( k_{x_+} \left(T_{i+1}^n + T_{i+1}^{n+1} - T_i^n - T_i^{n+1}\right) - k_{x_-} \left(T_i^n + T_i^{n+1} - T_{i-1}^n - T_{i-1}^{n+1}\right)\right) + q
\end{multline}
We multiply by $\frac{\Delta t}{\VHC}$ and move the terms with time step $n+1$ to the left hand side:
\begin{equation}
\begin{split}
- \frac{\Delta t k_{x_+}}{2\VHC\Delta x^2} T_{i+1}^{n+1} + \left( 1 + \frac{\Delta t (k_{x_+}+k_{x_-})}{2\VHC\Delta x^2} \right) T_i^{n+1} - \frac{\Delta t k_{x_-}}{2\VHC\Delta x^2} T_{i-1}^{n+1}\\
= \frac{\Delta t k_{x_+}}{2\VHC\Delta x^2} T_{i+1}^n + \left( 1 - \frac{\Delta t (k_{x_+} + k_{x_-})}{2\VHC\Delta x^2} \right) T_i^n + \frac{\Delta t k_{x_-}}{2\VHC\Delta x^2} T_{i-1}^n + \frac{\Delta t}{\VHC} q
\end{split}
\end{equation}
We see that it is convenient to introduce $\alpha_{x_\pm} = \frac{\Delta t k_{x_\pm}}{2\VHC\Delta x^2}$ and $\beta = \frac{\Delta t}{\VHC} q$ so we can write
\begin{equation}
\begin{split}
-\alpha_{x_+} T_{i+1}^{n+1} + \left( 1 + \alpha_{x_+} + \alpha_{x_-} \right) T_i^{n+1} - \alpha_{x_-} T_{i-1}^{n+1} =\\ \alpha_{x_+} T_{i+1}^n + \left( 1 - \alpha_{x_+} - \alpha_{x_-} \right) T_i^n + \alpha_{x_-} T_{i-1}^n + \beta
\end{split}
\end{equation}

Representing the temperatures and the heat deposition rate as column vectors, we can write the equation in terms of tridiagonal matrix multiplication and column vector addition
\begin{equation} \label{LHSRHS}
M_\mathrm{LHS} T^{n+1} = M_\mathrm{RHS} T^n + \beta
\end{equation}
where the matrices for insulating boundaries are (example shown for a 5-element grid)
\begin{align}
&M_\mathrm{LHS} = \begin{bmatrix}
1 + \alpha_{x_+} & -\alpha_{x_+} & 0 & 0 & 0\\
-\alpha_{x_-} & 1 + \alpha_{x_+} + \alpha_{x_-} & -\alpha_{x_+} & 0 & 0\\
0 & -\alpha_{x_-} & 1 + \alpha_{x_+} + \alpha_{x_-} & -\alpha_{x_+} & 0\\
0 & 0 & -\alpha_{x_-} & 1 + \alpha_{x_+} + \alpha_{x_-} & -\alpha_{x_+}\\
0 & 0 & 0 & -\alpha_{x_-} & 1 + \alpha_{x_-}
\end{bmatrix}\\
&M_\mathrm{RHS} = \begin{bmatrix}
1 - \alpha_{x_+} & \alpha_{x_+} & 0 & 0 & 0\\
\alpha_{x_-} & 1 - \alpha_{x_+} - \alpha_{x_-} & \alpha_{x_+} & 0 & 0\\
0 & \alpha_{x_-} & 1 - \alpha_{x_+} - \alpha_{x_-} & \alpha_{x_+} & 0\\
0 & 0 & \alpha_{x_-} & 1 - \alpha_{x_+} - \alpha_{x_-} & \alpha_{x_+}\\
0 & 0 & 0 & \alpha_{x_-} & 1 - \alpha_{x_-}
\end{bmatrix}
\end{align}
in which the $\alpha_{x_\pm}$ parameters in the $i$'th row in both matrices are to be evaluated at the spatial position of the $i$'th pixel/voxel.

The matrices for heat sinked boundaries are
\begin{align}
&M_\mathrm{LHS} = \begin{bmatrix}
1 & 0 & 0 & 0 & 0\\
-\alpha_{x_-} & 1 + \alpha_{x_+} + \alpha_{x_-} & -\alpha_{x_+} & 0 & 0\\
0 & -\alpha_{x_-} & 1 + \alpha_{x_+} + \alpha_{x_-} & -\alpha_{x_+} & 0\\
0 & 0 & -\alpha_{x_-} & 1 + \alpha_{x_+} + \alpha_{x_-} & -\alpha_{x_+}\\
0 & 0 & 0 & 0 & 1
\end{bmatrix}\\
&M_\mathrm{RHS} = \begin{bmatrix}
1 & 0 & 0 & 0 & 0\\
\alpha_{x_-} & 1 - \alpha_{x_+} - \alpha_{x_-} & \alpha_{x_+} & 0 & 0\\
0 & \alpha_{x_-} & 1 - \alpha_{x_+} - \alpha_{x_-} & \alpha_{x_+} & 0\\
0 & 0 & \alpha_{x_-} & 1 - \alpha_{x_+} - \alpha_{x_-} & \alpha_{x_+}\\
0 & 0 & 0 & 0 & 1
\end{bmatrix}
\end{align}
where furthermore the edge values of $\beta$ must be set to zero.

To calculate the temperature $T^{n+1}$, one can calculate first the right hand side of Eq. \ref{LHSRHS} by simple matrix multiplication and addition and then solve the remaining system of linear equations for $T^{n+1}$ using, for example, the "matrix left division" of MATLAB.



\section{Discretization in 2D}
Similar to Eq. \ref{1Dcont}, we can write the equation in 2D
\begin{equation}
\VHC \frac{dT}{dt} = \frac d {dx} \left( k \frac d {dx} T \right) + \frac d {dy} \left( k \frac d {dy} T \right) + q
\end{equation}
We discretize and insert an intermediate time step, denoted by the superscript $n + \frac 1 2$, and we use the approach of Douglas and use different temperature estimates for the different operators in the two steps. $\Delta t$ is the time difference between sub-steps, that is, the time difference between $n$ and $n+1$ is $2\Delta t$. The equation for the step from $n$ to $n + \frac 1 2$ is
\begin{equation}
\VHC \frac{\Delta T}{\Delta t} = \frac{\Delta_x}{\Delta x} \left( k \frac{\Delta_x}{2\Delta x} \left( T^n + T^{n+\frac 1 2} \right) \right) + \frac{\Delta_y}{\Delta y} \left( k \frac{\Delta_y}{\Delta y} T^n \right) + q
\end{equation}
and the equation for the step from $n + \frac 1 2$ to $n+1$ is
\begin{equation}
\VHC \frac{\Delta T}{\Delta t} = \frac{\Delta_x}{\Delta x} \left( k \frac{\Delta_x}{2\Delta x} \left( T^n + T^{n+\frac 1 2} \right) \right) + \frac{\Delta_y}{\Delta y} \left( k \frac{\Delta_y}{2\Delta y} \left( T^n + T^{n+1} \right) \right) + q
\end{equation}
Introducing $\alpha$ and $\beta$ as in the 1D case, we get for the first step
\begin{multline}
-\alpha_{x_+} T_{i+1,j}^{n+ \frac 1 2} + \left( 1 + \alpha_{x_+} + \alpha_{x_-} \right) T_{i,j}^{n+\frac 1 2} - \alpha_{x_-} T_{i-1,j}^{n+\frac 1 2} =\\ \alpha_{x_+} T_{i+1,j}^n + 2\alpha_{y_+} T_{i,j+1}^n + \left( 1 - \alpha_{x_+} - \alpha_{x_-} - 2\alpha_{y_+} - 2\alpha_{y_-} \right) T_{i,j}^n + \alpha_{x_-} T_{i-1,j}^n + 2\alpha_{y_-} T_{i,j-1}^n + \beta
\end{multline}
Writing this in matrix form we get (example shown for grid with size 3 in the x direction and size 3 in the y direction)
\begin{landscape}
\begin{equation}
M_\mathrm{LHS} = \left[ \begin{smallmatrix}
1 + \alpha_{x_+} & -\alpha_{x_+}                   & 0                & 0 & 0 & 0 & 0 & 0 & 0\\
-\alpha_{x_-}    & 1 + \alpha_{x_+} + \alpha_{x_-} & -\alpha_{x_+}    & 0 & 0 & 0 & 0 & 0 & 0\\
0                & -\alpha_{x_-}                   & 1 + \alpha_{x_-} & 0 & 0 & 0 & 0 & 0 & 0\\
0 & 0 & 0 & 1 + \alpha_{x_+} & -\alpha_{x_+}                   & 0                & 0 & 0 & 0\\
0 & 0 & 0 & -\alpha_{x_-}    & 1 + \alpha_{x_-} + \alpha_{x_+} & -\alpha_{x_+}    & 0 & 0 & 0\\
0 & 0 & 0 & 0                & -\alpha_{x_-}                   & 1 + \alpha_{x_-} & 0 & 0 & 0\\
0 & 0 & 0 & 0 & 0 & 0 & 1 + \alpha_{x_+}  & -\alpha_{x_+}                   & 0\\
0 & 0 & 0 & 0 & 0 & 0 & -\alpha_{x_-}     & 1 + \alpha_{x_+} + \alpha_{x_-} & -\alpha_{x_+}\\
0 & 0 & 0 & 0 & 0 & 0 & 0                 & -\alpha_{x_-}                   & 1 + \alpha_{x_-}             
\end{smallmatrix} \right]
\end{equation}
\begin{multline}
M_\mathrm{RHS} = \\
\left[ \begin{smallmatrix}
1 - \alpha_{x_+} - 2\alpha_{y_+} & \alpha_{x_+}                                   & 0                               & 2\alpha_{y_+} & 0 & 0 & 0 & 0 & 0\\
\alpha_{x_-}                    & 1 - \alpha_{x_+} - \alpha_{x_-} - 2\alpha_{y_+} & \alpha_{x_+}                    & 0 & 2\alpha_{y_+} & 0 & 0 & 0 & 0\\
0                               & \alpha_{x_-}                                   & 1 - \alpha_{x_-} - 2\alpha_{y_+} & 0 & 0 & 2\alpha_{y_+} & 0 & 0 & 0\\
2\alpha_{y_-} & 0 & 0 & 1 - \alpha_{x_+} - 2\alpha_{y_-} - 2\alpha_{y_+} & \alpha_{x_+}                                                  & 0 & 2\alpha_{y_+} & 0 & 0\\
0 & 2\alpha_{y_-} & 0 & \alpha_{x_-}                                   & 1 - \alpha_{x_-} - \alpha_{x_+} - 2\alpha_{y_-} - 2\alpha_{y_+} & \alpha_{x_+} & 0 & 2\alpha_{y_+} & 0\\
0 & 0 & 2\alpha_{y_-} & 0                                              & \alpha_{x_-}                                                  & 1 - \alpha_{x_-} - 2\alpha_{y_-} - 2\alpha_{y_+} & 0 & 0 & 2\alpha_{y_+}\\
0 & 0 & 0 & 2\alpha_{y_-} & 0 & 0 & 1 - \alpha_{x_+} - 2\alpha_{y_+} & \alpha_{x_+}                   & 0\\
0 & 0 & 0 & 0 & 2\alpha_{y_-} & 0 & \alpha_{x_-}      & 1 - \alpha_{x_+} - \alpha_{x_-} - 2\alpha_{y_+} & \alpha_{x_+}\\
0 & 0 & 0 & 0 & 0 & 2\alpha_{y_-} & 0                 & \alpha_{x_-}                   & 1 - \alpha_{x_-} - 2\alpha_{y_+}
\end{smallmatrix} \right]
\end{multline}
\end{landscape}

Before the second step, we transpose the grid (in practice rearranging the ordering of the data in the PCs memory representation), which I will show in the following by having the subscript $j$ index (which still corresponds to the $y$ direction) first and the $i$ index last.
\begin{multline}
-\alpha_{y_+} T_{j+1,i}^{n+1} + \left( 1 + \alpha_{y_+} + \alpha_{y_-} \right) T_{j,i}^{n+1} - \alpha_{y_-} T_{j-1,i}^{n+1} =\\ \alpha_{x_+} T_{j,i+1}^n + \alpha_{y_+} T_{j+1,i}^n + \left( - \alpha_{x_+} - \alpha_{x_-} - \alpha_{y_+} - \alpha_{y_-} \right) T_{j,i}^n + \alpha_{x_-} T_{j,i-1}^n + \alpha_{y_-} T_{j-1,i}^n +\\
\alpha_{x_+} T_{j,i+1}^{n+\frac 1 2} + \left( 1 - \alpha_{x_+} - \alpha_{x_-} \right) T_{j,i}^{n+\frac 1 2} + \alpha_{x_-} T_{j,i-1}^{n+\frac 1 2} + \beta
\end{multline}

\end{document}







