%\documentclass[fleqn]{scrartcl}
\documentclass{scrartcl}
\usepackage{amsmath}
%\setlength{\mathindent}{0pt}
\newcommand{\VHC}{c_\mathrm{VHC}}


\title{Theory for derivation of the expressions for Douglas-Gunn Alternating Direction Implicit (DG-ADI) method for solving the inhomogeneous heat diffusion equation on a rectangular cuboid finite element grid}
\author{Anders K. Hansen}

\begin{document}

\maketitle

\section{Heat equation}
$t$ is the time. The remaining quantities are considered functions of space: The volumetric heat capacity $\VHC$, with units of $\mathrm{\frac J {m^3 K}}$, the temperature $T$ with units of $\mathrm K$, the rate of heat deposition per unit volume per unit time $q$ with units of $\mathrm{\frac W {m^3 s}}$ and the thermal conductivity $k$, with units of $\mathrm{\frac W {m \cdot K}}$.

The heat diffusion equation is
\begin{equation}
\VHC \frac{dT}{dt} = \nabla \cdot \left(k \nabla T\right) + q
\end{equation}

\section{Discretization in 1D}
Let $k_{x+}$ denote the effective thermal conductivity for heat flow between a voxel and its neighbor in the plus x direction: $k_{x_+} = \frac{2k_ik_{i+1}}{k_i + k_{i+1}}$. Similarly, $k_{x_-} = \frac{2k_{i-1}k_i}{k_{i-1} + k_i}$.

Writing the equation in 1D yields
\begin{equation}
\VHC \frac{dT}{dt} = \frac d {dx} \left( k \frac d {dx} T \right) + q
\end{equation}
Using the notation $T^n$ to denote the temperature at time step $n$, we can discretize the above as
\begin{equation}
\VHC \frac{\Delta T}{\Delta t} = \frac \Delta {\Delta x} \left( k \frac \Delta {2\Delta x} \left( T^n + T^{n+1} \right) \right) + q
\end{equation}
where we have chosen to use the mean $\frac 1 2 \left( T^n + T^{n+1} \right)$ as the estimate of the temperature.

We can expand this to get
\begin{multline}
\VHC \frac{T^{n+1}-T^n}{\Delta t} = \frac 1 {2\Delta x^2} \Delta \left( k \Delta \left(T^n + T^{n+1} \right) \right) + q\\
= \frac 1 {2\Delta x^2} \left( k_{x_+} \left(T_{i+1}^n + T_{i+1}^{n+1} - T_i^n - T_i^{n+1}\right) - k_{x_-} \left(T_i^n + T_i^{n+1} - T_{i-1}^n - T_{i-1}^{n+1}\right)\right) + q
\end{multline}
Moving the terms with time step $n+1$ to the left hand side:
\begin{equation}
\begin{split}
- \frac{k_{x_+}}{2\Delta x^2} T_{i+1}^{n+1} + \left( \frac \VHC {\Delta t} + \frac{k_{x_+}+k_{x_-}}{2\Delta x^2} \right) T_i^{n+1} - \frac{k_{x_-}}{2\Delta x^2} T_{i-1}^{n+1}\\
= \frac{k_{x_+}}{2\Delta x^2} T_{i+1}^n + \left( \frac \VHC {\Delta t} - \frac{k_{x_+} + k_{x_-}}{2\Delta x^2} \right) T_i^n + \frac{k_{x_-}}{2\Delta x^2} T_{i-1}^n + q
\end{split}
\end{equation}
Representing the temperatures and the heat deposition rate as column vectors, we can write the equation in terms of tridiagonal matrix multiplication and column vector addition
\begin{equation} \label{LHSRHS}
M_\mathrm{LHS} T^{n+1} = M_\mathrm{RHS} T^n + q
\end{equation}
where the matrices for insulating boundaries are (example shown for a 5-element grid)
\begin{align}
&M_\mathrm{LHS} = \begin{bmatrix}
\frac \VHC {\Delta t} + \frac{k_{x_+}}{2\Delta x^2} & - \frac{k_{x_+}}{2\Delta x^2} & 0 & 0 & 0\\
- \frac{k_{x_-}}{2\Delta x^2} & \frac \VHC {\Delta t} + \frac{k_{x_+}+k_{x_-}}{2\Delta x^2} & - \frac{k_{x_+}}{2\Delta x^2} & 0 & 0\\
0 & - \frac{k_{x_-}}{2\Delta x^2} & \frac \VHC {\Delta t} + \frac{k_{x_+}+k_{x_-}}{2\Delta x^2} & - \frac{k_{x_+}}{2\Delta x^2} & 0\\
0 & 0 & - \frac{k_{x_-}}{2\Delta x^2} & \frac \VHC {\Delta t} + \frac{k_{x_+}+k_{x_-}}{2\Delta x^2} & - \frac{k_{x_+}}{2\Delta x^2}\\
0 & 0 & 0 & - \frac{k_{x_-}}{2\Delta x^2} & \frac \VHC {\Delta t} + \frac{k_{x_-}}{2\Delta x^2}
\end{bmatrix}\\
&M_\mathrm{RHS} = \begin{bmatrix}
\frac \VHC {\Delta t} - \frac{k_{x_+}}{2\Delta x^2} & \frac{k_{x_+}}{2\Delta x^2} & 0 & 0 & 0\\
\frac{k_{x_-}}{2\Delta x^2} & \frac \VHC {\Delta t} - \frac{k_{x_+}+k_{x_-}}{2\Delta x^2} & \frac{k_{x_+}}{2\Delta x^2} & 0 & 0\\
0 & \frac{k_{x_-}}{2\Delta x^2} & \frac \VHC {\Delta t} - \frac{k_{x_+}+k_{x_-}}{2\Delta x^2} & \frac{k_{x_+}}{2\Delta x^2} & 0\\
0 & 0 & \frac{k_{x_-}}{2\Delta x^2} & \frac \VHC {\Delta t} - \frac{k_{x_+}+k_{x_-}}{2\Delta x^2} & \frac{k_{x_+}}{2\Delta x^2}\\
0 & 0 & 0 & \frac{k_{x_-}}{2\Delta x^2} & \frac \VHC {\Delta t} - \frac{k_{x_-}}{2\Delta x^2}
\end{bmatrix}
\end{align}
in which $\VHC$, $k_{x_+}$, and $k_{x_-}$ in the $i$'th row in both matrices are to be evaluated at the spatial position of the $i$'th pixel/voxel.

The matrices for heat sinked boundaries are
\begin{align}
&M_\mathrm{LHS} = \begin{bmatrix}
1 & 0 & 0 & 0 & 0\\
- \frac{k_{x_-}}{2\Delta x^2} & \frac \VHC {\Delta t} + \frac{k_{x_+}+k_{x_-}}{2\Delta x^2} & - \frac{k_{x_+}}{2\Delta x^2} & 0 & 0\\
0 & - \frac{k_{x_-}}{2\Delta x^2} & \frac \VHC {\Delta t} + \frac{k_{x_+}+k_{x_-}}{2\Delta x^2} & - \frac{k_{x_+}}{2\Delta x^2} & 0\\
0 & 0 & - \frac{k_{x_-}}{2\Delta x^2} & \frac \VHC {\Delta t} + \frac{k_{x_+}+k_{x_-}}{2\Delta x^2} & - \frac{k_{x_+}}{2\Delta x^2}\\
0 & 0 & 0 & 0 & 1
\end{bmatrix}\\
&M_\mathrm{RHS} = \begin{bmatrix}
1 & 0 & 0 & 0 & 0\\
\frac{k_{x_-}}{2\Delta x^2} & \frac \VHC {\Delta t} - \frac{k_{x_+}+k_{x_-}}{2\Delta x^2} & \frac{k_{x_+}}{2\Delta x^2} & 0 & 0\\
0 & \frac{k_{x_-}}{2\Delta x^2} & \frac \VHC {\Delta t} - \frac{k_{x_+}+k_{x_-}}{2\Delta x^2} & \frac{k_{x_+}}{2\Delta x^2} & 0\\
0 & 0 & \frac{k_{x_-}}{2\Delta x^2} & \frac \VHC {\Delta t} - \frac{k_{x_+}+k_{x_-}}{2\Delta x^2} & \frac{k_{x_+}}{2\Delta x^2}\\
0 & 0 & 0 & 0 & 1
\end{bmatrix}
\end{align}
where furthermore the edge values of $q$ must be zero.

To calculate the temperature $T^{n+1}$, one can calculate first the right hand side of Eq. \ref{LHSRHS} by simple matrix multiplication and addition and then solve the remaining system of linear equations for $T^{n+1}$ using, for example, the "matrix left division" of MATLAB.

\section{Discretization in 2D}


\end{document}







